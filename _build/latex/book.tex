%% Generated by Sphinx.
\def\sphinxdocclass{jupyterBook}
\documentclass[letterpaper,10pt,italian]{jupyterBook}
\ifdefined\pdfpxdimen
   \let\sphinxpxdimen\pdfpxdimen\else\newdimen\sphinxpxdimen
\fi \sphinxpxdimen=.75bp\relax
\ifdefined\pdfimageresolution
    \pdfimageresolution= \numexpr \dimexpr1in\relax/\sphinxpxdimen\relax
\fi
%% let collapsible pdf bookmarks panel have high depth per default
\PassOptionsToPackage{bookmarksdepth=5}{hyperref}
%% turn off hyperref patch of \index as sphinx.xdy xindy module takes care of
%% suitable \hyperpage mark-up, working around hyperref-xindy incompatibility
\PassOptionsToPackage{hyperindex=false}{hyperref}
%% memoir class requires extra handling
\makeatletter\@ifclassloaded{memoir}
{\ifdefined\memhyperindexfalse\memhyperindexfalse\fi}{}\makeatother

\PassOptionsToPackage{warn}{textcomp}

\catcode`^^^^00a0\active\protected\def^^^^00a0{\leavevmode\nobreak\ }
\usepackage{cmap}
\usepackage{fontspec}
\defaultfontfeatures[\rmfamily,\sffamily,\ttfamily]{}
\usepackage{amsmath,amssymb,amstext}
\usepackage{polyglossia}
\setmainlanguage{italian}



\setmainfont{FreeSerif}[
  Extension      = .otf,
  UprightFont    = *,
  ItalicFont     = *Italic,
  BoldFont       = *Bold,
  BoldItalicFont = *BoldItalic
]
\setsansfont{FreeSans}[
  Extension      = .otf,
  UprightFont    = *,
  ItalicFont     = *Oblique,
  BoldFont       = *Bold,
  BoldItalicFont = *BoldOblique,
]
\setmonofont{FreeMono}[
  Extension      = .otf,
  UprightFont    = *,
  ItalicFont     = *Oblique,
  BoldFont       = *Bold,
  BoldItalicFont = *BoldOblique,
]



\usepackage[Sonny]{fncychap}
\ChNameVar{\Large\normalfont\sffamily}
\ChTitleVar{\Large\normalfont\sffamily}
\usepackage[,numfigreset=1,mathnumfig]{sphinx}

\fvset{fontsize=\small}
\usepackage{geometry}


% Include hyperref last.
\usepackage{hyperref}
% Fix anchor placement for figures with captions.
\usepackage{hypcap}% it must be loaded after hyperref.
% Set up styles of URL: it should be placed after hyperref.
\urlstyle{same}


\usepackage{sphinxmessages}



        % Start of preamble defined in sphinx-jupyterbook-latex %
         \usepackage[Latin,Greek]{ucharclasses}
        \usepackage{unicode-math}
        % fixing title of the toc
        \addto\captionsenglish{\renewcommand{\contentsname}{Contents}}
        \hypersetup{
            pdfencoding=auto,
            psdextra
        }
        % End of preamble defined in sphinx-jupyterbook-latex %
        

\title{basics blog}
\date{24 nov 2025}
\release{}
\author{basics}
\newcommand{\sphinxlogo}{\vbox{}}
\renewcommand{\releasename}{}
\makeindex
\begin{document}

\pagestyle{empty}
\sphinxmaketitle
\pagestyle{plain}
\sphinxtableofcontents
\pagestyle{normal}
\phantomsection\label{\detokenize{index::doc}}



\bigskip\hrule\bigskip


\begin{DUlineblock}{0em}
\item[] \sphinxstylestrong{\Large Latest Posts}
\end{DUlineblock}


\bigskip\hrule\bigskip


\begin{DUlineblock}{0em}
\item[] \sphinxstylestrong{{\hyperref[\detokenize{posts/2025-11-12-structures-and-weak-form::doc}]{\sphinxcrossref{\DUrole{doc,std,std-doc}{Weak form of equations and «energy» theorems in structural mechanics}}}}}
\end{DUlineblock}

\sphinxAtStartPar
\sphinxstyleemphasis{2025\sphinxhyphen{}11\sphinxhyphen{}12}

\begin{DUlineblock}{0em}
\item[] \sphinxstylestrong{{\hyperref[\detokenize{posts/2025-06-09-my-first-post::doc}]{\sphinxcrossref{\DUrole{doc,std,std-doc}{My First Post}}}}}
\end{DUlineblock}

\sphinxAtStartPar
\sphinxstyleemphasis{2025\sphinxhyphen{}06\sphinxhyphen{}09}

\sphinxAtStartPar
bla bla bla
bla bla bla
bla bla bla
bla bla bla

\sphinxstepscope

\sphinxAtStartPar
\sphinxincludegraphics{{my-first-post}.png}


\chapter{My first post}
\label{\detokenize{posts/2025-06-09-my-first-post:my-first-post}}\label{\detokenize{posts/2025-06-09-my-first-post:blog-first-post}}\label{\detokenize{posts/2025-06-09-my-first-post::doc}}
\sphinxAtStartPar
bla bla bla
bla bla bla
bla bla bla
bla bla bla
bla bla bla
bla bla bla
bla bla bla
bla bla bla
bla bla bla
bla bla bla
bla bla bla
bla bla bla
bla bla bla
bla bla bla

\sphinxstepscope

\sphinxAtStartPar
\sphinxincludegraphics{{s03}.png}


\chapter{Weak form of equations and «energy» theorems in structural mechanics}
\label{\detokenize{posts/2025-11-12-structures-and-weak-form:weak-form-of-equations-and-energy-theorems-in-structural-mechanics}}\label{\detokenize{posts/2025-11-12-structures-and-weak-form:blog-structures-weak-eqn}}\label{\detokenize{posts/2025-11-12-structures-and-weak-form::doc}}
\sphinxAtStartPar
Questo post discute il ruolo della forma debole delle equazioni nell’ambito della meccanica strutturale.







\renewcommand{\indexname}{Indice}
\printindex
\end{document}